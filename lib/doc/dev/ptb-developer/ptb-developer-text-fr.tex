%_______________________________________________________________________________
\section{Manipulation de texte}

\subsection{L'espace de noms text \index{text!espace de noms} }
Tout ce qui se rapporte au traitement de textes est dans l'espace de noms
\texttt{text}.

\subsection{Polices de caract�res \index{text!police} }
Les polices de caract�res sont repr�sent�es par la classe \verb|text::font|. Il
faut passer l'image des caract�res au constructeur, ainsi que la taille des
caract�res. Toutes les polices sont � chasse fixe.

L'image doit contenir, dans l'ordre : les minuscules, les majuscules,
les chiffres, l'espace et les caract�res sp�ciaux :

\begin{verbatim}
! " # $ % & \ ( ) *
+ , - . / : ; < = >
? @ [ \ ] ^ _ ` { |
} ~ � � � � � � � �
� � � � � �
\end{verbatim}

\subsection{Mesures \index{text!mesure} }
La classe \verb|text::text_metric| permet de mesurer la taille d'un texte
associ� � une police, en nombre de caract�res ou de pixels.

