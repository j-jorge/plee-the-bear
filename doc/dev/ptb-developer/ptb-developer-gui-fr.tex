%_______________________________________________________________________________
\section{Interface utilisateur}

%-------------------------------------------------------------------------------
\subsection{L'espace de nom gui \index{gui!espace de noms} }
Tout ce qui se rapporte � l'interface utilisateur est dans l'espace de noms
\texttt{gui}.

\subsection{Composant visuel de base
            \index{gui!composant visuel} }

Tous les composants visuels h�ritent de la classe \verb|gui::visual_component|.
La plupart des composants sont contenus dans un composant parent. C'est ce
composant qui se chargera de d�truire ses composants fils lors de sa
destruction. Un composant a une taille et une position relative au coin en
haut � gauche du composant parent. Les composants fils sont enti�rement inclus
(au sens g�om�trique) dans le composant parent (pas de d�bordement).

Chaque composant se charge d'afficher ses fils (c'est automatique). Les fils
peuvent red�finir la m�thode \texttt{display()} pour s'afficher. Il n'y a
aucune garantie sur l'ordre d'affichage des composants fils.

\subsection{Les fen�tres \index{gui!fenetre@fen�tre} }

La fen�tre est le premier composant graphique. Elle est repr�sent�e par la
classe \verb|gui::window|. Il s'agit d'un cadre avec un fond et un bord,
destin� � contenir des contr�les. Il faut indiquer au constructeur quels sont
les sprites � utiliser pour le fond, les bords et les coins. Les sprites de
fond et de bords sont r�p�t�s pour remplir l'espace.

Actuellement l'affichage n'affiche que les morceaux entiers de l'arri�re plan
(pas de coupure sur les bords).

\subsection{Zone de texte en lecture seule
            \index{gui!zone de texte} }

Il s'agit d'un composant simple, charg� d'afficher du texte � l'�cran. Il est
repr�sent� par la classe \verb|gui::static_text|. La taille peut �tre ajust�e
automatiquement au texte, autrement il sera coup� au niveau des espaces.

Il faut indiquer au contructeur quelle est la police � utiliser pour afficher
le texte. Cette police sera d�truite en m�me temps que le composant.

\subsection{Zone de texte en lecture seule et d�filable
            \index{gui!zone de texte defilable@zone de texte d�filable} }

Cette classe (\verb|gui::multi_page|) permet d'afficher un texte qui ne 
tiendrait pas � l'�cran en permettant de faire d�filer le texte.

\subsection{Zone de saisie \index{gui!zone de saisie} }

La classe \verb|gui::text_input| permet de r�cup�rer une ligne de texte entr�e
par l'utilisateur.

